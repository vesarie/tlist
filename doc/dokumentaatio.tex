\documentclass[a4paper,english,oneside,titlepage]{article}
%\documentclass[a4paper,english,twocolumn,oneside,titlepage]{article}
%\documentclass[a4paper,twoside,english,twocolumn,oneside]{report}

\usepackage[utf8]{inputenc}
\usepackage[T1]{fontenc}

% % add 'finnish' to get labels like 'Taulukko' etc. instead of 'Table'
%\usepackage[english,finnish]{babel}
\usepackage[finnish]{babel}

% Enable/disable MathTime fonts depending on if they are available
%\usepackage[T1,mtbold,lucidacal,mtplusscr,subscriptcorrection]{mathtime}

\usepackage{times}
\usepackage[pdftex]{graphicx}
\usepackage{color}
\usepackage[pdftex,colorlinks=true,citecolor=black,
            pagecolor=black,linkcolor=black,menucolor=black,
            urlcolor=black]{hyperref}
\usepackage{eufrak}
\usepackage{amsmath}
\usepackage{amsbsy}
\usepackage{eucal}
%\usepackage{subfigure}

\usepackage{longtable}
\usepackage{url}
\urlstyle{same}

\usepackage{booktabs}

% For pmatrix* (starred version which allows to specify alignment)
\usepackage{mathtools}

%
%\usepackage{natbib}
%
\usepackage[]{natbib}
\usepackage{graphicx,enumerate}
\bibliographystyle{plain}

\usepackage{url}
%% Define a new 'leo' style for the package that will use a smaller font.
\makeatletter
\def\url@leostyle{%
  \@ifundefined{selectfont}{\def\UrlFont{\sf}}{\def\UrlFont{\small\ttfamily}}}
\makeatother
%% Now actually use the newly defined style.
%\urlstyle{leo}

\usepackage{multicol}

% Euro-merkki
%\usepackage{textcomp}
%\usepackage[official]{eurosym}
\usepackage[gen]{eurosym}

% Theorem env: http://www.maths.tcd.ie/~dwilkins/LaTeXPrimer/Theorems.html
\newtheorem{theorem}{Theorem}[section]
\newtheorem{lemma}[theorem]{Lemma}
\newtheorem{proposition}[theorem]{Proposition}
\newtheorem{corollary}[theorem]{Corollary}

\newtheorem{remark}[theorem]{Remark}
\newtheorem{example}[theorem]{Example}
\newtheorem{definition}[theorem]{Definition}

\newenvironment{proof}[1][Proof]{\begin{trivlist}
\item[\hskip \labelsep {\bfseries #1}]}{\end{trivlist}}
%\newenvironment{definition}[1][Definition]{\begin{trivlist}
%\item[\hskip \labelsep {\bfseries #1}]}{\end{trivlist}}
%\newenvironment{example}[1][Example]{\begin{trivlist}
%\item[\hskip \labelsep {\bfseries #1}]}{\end{trivlist}}
%\newenvironment{remark}[1][Remark]{\begin{trivlist}
%\item[\hskip \labelsep {\bfseries #1}]}{\end{trivlist}}

\newenvironment{motivation}[1][Motivation]{\begin{trivlist}
\item[\hskip \labelsep {\bfseries #1}]}{\end{trivlist}}

\newcommand{\qed}{\nobreak \ifvmode \relax \else
      \ifdim\lastskip<1.5em \hskip-\lastskip
      \hskip1.5em plus0em minus0.5em \fi \nobreak
      \vrule height0.75em width0.5em depth0.25em\fi}
% end of theorem env

% declare ess sup operator
\DeclareMathOperator{\Esssup}{ess\,sup}

% abbreviate Ito
\newcommand{\Ito}{It\^{o}}

\newcommand{\thedate}{\today}

% Commands for common sums
\newcommand{\Sumij}{\sum_{i,j=1}^n}
\newcommand{\Sumi}{\sum_{i=1}^n}

% Roman numerals
\makeatletter
\newcommand{\rmnum}[1]{\romannumeral #1}
\newcommand{\Rmnum}[1]{\expandafter\@slowromancap\romannumeral #1@}
\makeatother

% Command for the string: cubic meters per hectare - m3/ha
\newcommand{\cmha}{m$^3$/ha}

% Try to prevent widow and orphan lines
\widowpenalty=300
\clubpenalty=300
\setlength{\parskip}{3ex plus 2ex minus 2ex}

%\setlength{\parskip}{1em}
\setlength{\parindent}{0em}

% SUOMI: Poikkeustavutuslista
\hyphenation{  }


% Headings using the fancyhdr package
\usepackage{fancyhdr}
\setlength{\headheight}{15pt}
\pagestyle{fancy}

\fancyhf{}
 
\lhead{Aineopintojen harjoitustyö: Tietokantasovellus}
\rhead{Vesa Riekkinen}
\rfoot{\thepage}

\pdfinfo{
          /Title      (582203 Aineopintojen harjoitustyö: Tietokantasovellus)
          /Author     (Vesa Riekkinen)
          /Keywords   ()
}

\title{Tehtävälista}
\date{21.3.2016}
\author{582203 Aineopintojen harjoitustyö: Tietokantasovellus\\ \\ Vesa Riekkinen\\ \\Helsingin yliopisto\\Matemaattis-luonnontieteellinen tiedekunta\\Tietojenkäsittelytiede}

%-------------------------------------------------------------------------------------------------------------------

\begin{document}

\maketitle

%\twocolumn
\setcounter{page}{1}
\pagenumbering{arabic}

\section{Johdanto}

Työn aihe on verkossa toimiva tehtävälista eli niin sanottu to do -lista. Sovelluksen käyttötarkoitus on, että käyttäjä voi kirjata sovellukseen tulevien päivien ja viikkojen tehtävät. Tämän tyyppisen sovelluksen tavoite on auttaa käyttäjiä heidän oman aikansa hallinnassa. Yksi perustavanlaatuinen syy tehtävälistan ylläpidolle on vähentää tehtävien järjestämisestä syntyvää aivojen kuormitusta kirjoittamalla ne esimerkiksi muistilapulle. Nykypäivänä lienee varsin luonnollista, että paperille kirjoitettujen muistilappujen lisäksi käytetään myös digitaalisia verkkopalveluita. Tämän tyyppisiä olemassa olevia verkkopalveluita ovat esimerkiksi todoist.com ja any.do.

Työn aihe on mukautettu versio valmiista aiheesta nimeltä Muistilista. Olennaisin ero valmiiseen aiheeseen verrattuna on, että tämä sovellus järjestää tehtävät etupäässä päivämäärän mukaan. Valmis aihe Muistilista vuorostaan järjestää tehtävät etupäässä niiden prioriteetin mukaan. Tämä sovellus tulee todennäköisesti tarjoamaan myös mahdollisuuden tehtävien prioriteettien asettamiseen, mutta tämä tulee olemaan enemmänkin lisäominaisuus. Pääasiallinen tarkoitus on pitää kirjaa tehtävistä niiden suunnitellun suorituspäivämäärän mukaan.

Sovellus ryhmittelee tehtävät sillä tavalla, että kukin tehtävä kuuluu johonkin projektiin. Projektin sisällä tehtäviä voi ryhmitellä edelleen sillä tavalla, että yhdellä tehtävällä voi olla alitehtäviä. Joka tapauksessa kullakin tehtävällä on sille osoitettu suorituspäivämäärä, jota voi tarpeen mukaan myös päivittää. Tehtäviä voi tarkastella sekä projektikohtaisesti, että kootusti. Koottu näkymä näyttää tulevan päivän ja tulevan viikon tehtävät päiväkohtaisesti.

Sovellus toteutetaan Javalla käyttäen Java EE:n web-profiilia. Web-profiili tosin sisältää useita eri tekniikoita, joista tämä sovellus käyttää vain pientä osaa. Tämä sovellus käyttää nimenomaan Java Servlet ja JavaServer Pages (JSP) -tekniikoita. Näitä tekniikoita täydentämään on kehitetty myös useita erilaisia sovelluskehyksiä, mutta tässä työssä ei ole tarkoitus käyttää mitään tällaista sovelluskehystä.

Servlet-tekniikan käyttö edellyttää sitä tukevan HTTP-palvelimen käyttöä. Kevyemmän mallisia servlet container -palvelimia ovat esimerkiksi Apache Tomcat ja Jetty. Nämä palvelimet ovat myös sillä tavalla erityisiä, että niitä voidaan ajaa niin sanotussa embedded-tilassa. Embedded-tila tarkoittaa olennaisesti sitä, että jokaiselle sovellukselle osoitetaan oma paikallinen HTTP-palvelin. Tällöin yksi palvelinohjelmiston kopio suorittaa vain yhtä sovellusta. Tämä on vastakohta perinteiselle mallille, jossa yksi keskitetty sovelluspalvelin pyörittää suurta joukkoa erilaisia sovelluksia. Uudentyyppistä hajautettua mallia käytetään joissakin pilvipalveluissa, jotka tyypillisesti pilkkovat kunkin palvelinkoneen fyysisesti tarjoamat resurssit useampaan virtuaaliseen ympäristöön. Tässä työssä hyödynnetään erästä tällaista pilvipalvelua.

Sovelluksen julkinen versio toimitetaan kehityksen aikana säännöllisesti Heroku-pilvipalveluun. Sovelluksen kotisivu on verkossa osoitteessa tlist.herokuapp.com. Heroku noudattaa edellä kuvattua hajautettua mallia, jossa kukin sovellus käynnistää oman palvelimensa. Palvelinohjelmiston voi siis valita vapaasti, mutta sen tulee pystyä toimimaan embedded-tilassa. Tässä työssä käytetään Apache Tomcat -sovelluspalvelinta, mutta sovellus lienee yhtä lailla yhtäsopiva Jettyn kanssa.

Sovelluksen tietokannaksi on valittu PostgreSQL. PostgreSQL on kurssimateriaalissa suositeltu valinta. Heroku ei myöskään tarjoa muuta tietokantaa ilmaisversiossaan. Sovellus on tarkoitus toteuttaa SQL-standardien mukaisella yhteensopivalla tavalla, mutta kussakin tietokannassa on tietyt erityispiirteensä. Tällainen erityispiirre on esimerkiksi PostgreSQL:n serial-tyyppi tietokantataulujen pääavaimille. Toisin sanoen sovellus olisi todennäköisesti siirrettävissä toiseen tietokantatyyppiin kohtuullisen pienellä vaivalla, mutta sellaisenaan se toimii vain PostgreSQL-tietokannalla.

Sovelluksen käyttäjille verkossa näkyvät sivut toteutetaan HTML- ja CSS-tekniikoilla. Sivujen rakentamisessa hyödynnetään Bootstrap-kirjastoa, joka sisältää suuren joukon valmiita käyttöliittymäkomponentteja. Useimmat Bootstrapin komponentit ovat siinä mielessä staattisia, että ne eivät käytä Javascript-tekniikkaa. Tässä työssä on tarkoitus hyödyntää staattisten komponenttien lisäksi myös muutamaa Bootstrapin Javascript-komponenttia. Siten käyttäjän selaimen on tuettava myös Javascriptiä, mikä on modernien selainten perusominaisuus.



\end{document}


%%% Local Variables: 
%%% mode: latex
%%% TeX-master: t
%%% End: 
