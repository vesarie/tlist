% !TEX TS-program = pdflatexmk
\documentclass[12pt,a4paper,oneside,titlepage,pdftex]{article}
\usepackage[utf8]{inputenc}
\usepackage[T1]{fontenc}
%\usepackage[OT1]{fontenc}

\usepackage[english,finnish]{babel}

% Enable/disable MathTime fonts depending on if they are available
%\usepackage[T1,mtbold,lucidacal,mtplusscr,subscriptcorrection]{mathtime}

%\usepackage{times}
%\usepackage{fourier}
%\usepackage{lmodern}
%\usepackage{palatino}
%\usepackage{tgpagella}

\usepackage{hyphenat}

\usepackage[pdftex]{graphicx}
\usepackage{color}
\usepackage[pdftex,colorlinks=true,citecolor=black,
            pagecolor=black,linkcolor=black,menucolor=black,
            urlcolor=black]{hyperref}
\usepackage{eufrak}
\usepackage{amsmath}
\usepackage{amsbsy}
\usepackage{eucal}
%\usepackage{subfigure}

\usepackage{longtable}
\usepackage{url}
\urlstyle{same}

\usepackage{booktabs}
\usepackage{tabularx}
\usepackage{multicol}
\usepackage{multirow}
\usepackage{rotating}
\usepackage[margin=10pt,font=normalsize,labelfont=bf,labelsep=period]{caption}

% For pmatrix* (starred version which allows to specify alignment)
\usepackage{mathtools}

\usepackage[]{natbib}
\usepackage{graphicx,enumerate}
\bibliographystyle{plain}

\usepackage{url}
%% Define a new 'leo' style for the package that will use a smaller font.
\makeatletter
\def\url@leostyle{%
  \@ifundefined{selectfont}{\def\UrlFont{\sf}}{\def\UrlFont{\small\ttfamily}}}
\makeatother
%% Now actually use the newly defined style.
%\urlstyle{leo}

% Euro-merkki
%\usepackage{textcomp}
%\usepackage[official]{eurosym}
\usepackage[gen]{eurosym}

% Theorem env: http://www.maths.tcd.ie/~dwilkins/LaTeXPrimer/Theorems.html
\newtheorem{theorem}{Theorem}[section]
\newtheorem{lemma}[theorem]{Lemma}
\newtheorem{proposition}[theorem]{Proposition}
\newtheorem{corollary}[theorem]{Corollary}

\newtheorem{remark}[theorem]{Remark}
\newtheorem{example}[theorem]{Example}
\newtheorem{definition}[theorem]{Definition}

\newenvironment{proof}[1][Proof]{\begin{trivlist}
\item[\hskip \labelsep {\bfseries #1}]}{\end{trivlist}}
\newenvironment{motivation}[1][Motivation]{\begin{trivlist}
\item[\hskip \labelsep {\bfseries #1}]}{\end{trivlist}}

\newcommand{\qed}{\nobreak \ifvmode \relax \else
      \ifdim\lastskip<1.5em \hskip-\lastskip
      \hskip1.5em plus0em minus0.5em \fi \nobreak
      \vrule height0.75em width0.5em depth0.25em\fi}
% end of theorem env

\newcommand{\thedate}{\today}

% Commands for common sums
\newcommand{\Sumij}{\sum_{i,j=1}^n}
\newcommand{\Sumi}{\sum_{i=1}^n}

% Roman numerals
\makeatletter
\newcommand{\rmnum}[1]{\romannumeral #1}
\newcommand{\Rmnum}[1]{\expandafter\@slowromancap\romannumeral #1@}
\makeatother

% Try to prevent widow and orphan lines
\widowpenalty=300
\clubpenalty=300
\setlength{\parskip}{3ex plus 2ex minus 2ex}

%\setlength{\parskip}{1em}
\setlength{\parindent}{0em}

% SUOMI: Poikkeustavutuslista
\hyphenation{  }


% Headings using the fancyhdr package
\usepackage{fancyhdr}
\setlength{\headheight}{15pt}
\pagestyle{fancy}

\fancyhf{}
 
\lhead{Aineopintojen harjoitustyö: Tietokantasovellus}
\rhead{Vesa Riekkinen}
\rfoot{\thepage}

\pdfinfo{
          /Title      (582203 Aineopintojen harjoitustyö: Tietokantasovellus)
          /Author     (Vesa Riekkinen)
          /Keywords   ()
}

\usepackage[dmyyyy]{datetime}
\renewcommand{\dateseparator}{.}

\title{Tehtävälista}
\date{8.5.2016}
\author{582203 Aineopintojen harjoitustyö: Tietokantasovellus\\ \\ Vesa Riekkinen\\ \\Helsingin yliopisto\\Matemaattis-luonnontieteellinen tiedekunta\\Tietojenkäsittelytiede}

%-------------------------------------------------------------------------------------------------------------------

\begin{document}

\maketitle

\setcounter{page}{1}
\pagenumbering{arabic}
 
\section{Johdanto}

Työn aihe on verkossa toimiva tehtävälista eli niin sanottu to do -lista. Sovelluksen käyttötarkoitus on, että käyttäjä voi kirjata sovellukseen tulevien päivien ja viikkojen tehtävät. Tämän tyyppisen sovelluksen tavoite on auttaa käyttäjiä heidän oman aikansa hallinnassa. Yksi perustavanlaatuinen syy tehtävälistan ylläpidolle on vähentää tehtävien järjestämisestä syntyvää aivojen kuormitusta kirjoittamalla ne esimerkiksi muistilapulle. Nykypäivänä lienee varsin luonnollista, että paperille kirjoitettujen muistilappujen lisäksi käytetään myös digitaalisia verkkopalveluita. Tämän tyyppisiä olemassa olevia verkkopalveluita ovat esimerkiksi todoist.com ja any.do.

Työn aihe on mukautettu versio valmiista aiheesta nimeltä Muistilista. Olennaisin ero valmiiseen aiheeseen verrattuna on, että tämä sovellus järjestää tehtävät etupäässä päivämäärän mukaan. Valmis aihe Muistilista vuorostaan järjestää tehtävät etupäässä niiden prioriteetin mukaan. Sovellus tarjoaa mahdollisuuden myös prioriteettien asettamiseen, mutta tämä on enemmänkin lisäominaisuus. Pääasiallinen tarkoitus on pitää kirjaa tehtävistä niiden suunnitellun suorituspäivämäärän mukaan.

Sovellus on toteutettu Javalla käyttäen Java EE:n web-profiilia. Web-profiili tosin sisältää useita eri tekniikoita, joista tämä sovellus käyttää vain pientä osaa. Tämä sovellus käyttää nimenomaan Java Servlet ja JavaServer Pages (JSP) -tekniikoita. Näitä tekniikoita täydentämään on kehitetty myös useita erilaisia sovelluskehyksiä, mutta tässä työssä ei ole tarkoitus käyttää mitään tällaista sovelluskehystä.

Servlet-tekniikan käyttö edellyttää sitä tukevan HTTP-palvelimen käyttöä. Kevyemmän mallisia servlet container -palvelimia ovat esimerkiksi Apache Tomcat ja Jetty. Nämä palvelimet ovat myös siinä mielessä erityisiä, että niitä voidaan ajaa niin sanotussa embedded-tilassa. Embedded-tila tarkoittaa olennaisesti sitä, että jokaiselle sovellukselle osoitetaan oma paikallinen HTTP-palvelin. Tällöin yksi palvelinohjelmiston kopio suorittaa vain yhtä sovellusta. Tämä on vastakohta perinteiselle mallille, jossa yksi keskitetty sovelluspalvelin pyörittää suurta joukkoa erilaisia sovelluksia. Uudentyyppistä hajautettua mallia käytetään joissakin pilvipalveluissa, jotka tyypillisesti pilkkovat kunkin palvelinkoneen fyysisesti tarjoamat resurssit useampaan virtuaaliseen ympäristöön. Tässä työssä hyödynnetään erästä tällaista pilvipalvelua.

Sovelluksen julkinen versio on asennettu Heroku\hyp{}pilvipalveluun. Sovelluksen kotisivu on verkossa osoitteessa \url{http://tlist.herokuapp.com}. Heroku noudattaa edellä kuvattua hajautettua mallia, jossa kukin sovellus käynnistää oman palvelimensa. Palvelinohjelmiston voi siis valita vapaasti, mutta sen tulee pystyä toimimaan embedded-tilassa. Tässä työssä käytetään Apache Tomcat -sovelluspalvelinta, mutta sovellus lienee yhtä lailla yhtäsopiva Jettyn kanssa.

Sovelluksen tietokannaksi on valittu PostgreSQL. PostgreSQL on kurssimateriaalissa suositeltu valinta. Heroku ei myöskään tarjoa muuta tietokantaa ilmaisversiossaan. Sovellus on tarkoitus toteuttaa SQL-standardien mukaisella yhteensopivalla tavalla, mutta kussakin tietokannassa on tietyt erityispiirteensä. Tällainen erityispiirre on esimerkiksi PostgreSQL:n serial-tyyppi tietokantataulujen pääavaimille. Toisin sanoen sovellus olisi todennäköisesti siirrettävissä toiseen tietokantatyyppiin kohtuullisen pienellä vaivalla, mutta sellaisenaan se toimii vain PostgreSQL-tietokannalla.

Sovelluksen käyttäjille verkossa näkyvät sivut on toteutettu HTML-, CSS- ja JavaScript-tekniikoilla. Sivujen rakentamisessa on hyödynnetty Bootstrap-kirjastoa, joka sisältää suuren joukon valmiita käyttöliittymäkomponentteja. Useimmat Bootstrapin komponentit ovat siinä mielessä staattisia, että ne eivät käytä Javascript-tekniikkaa. Sivuston rakentamisessa on kuitenkin hyödynnetty staattisten komponenttien lisäksi myös jQuery-kirjaston AJAX-tukea. Siten käyttäjän selaimen on tuettava myös Javascriptiä, mikä on modernien selainten perusominaisuus.

\section{Yleiskuva järjestelmästä}

Sovellus ryhmittelee tehtävät sillä tavalla, että kukin tehtävä kuuluu yhteen tai useampaan projektiin. Tehtäviä voi tarkastella sekä projektikohtaisesti että kootusti esimerkiksi päivämäärän mukaan. Koottu näkymä näyttää esimerkiksi tulevan päivän ja tulevan viikon tehtävät päiväkohtaisesti. Tätä varten kullekin tehtävälle voidaan osoittaa suorituspäivämäärä ja prioriteetti.

Järjestelmä on tarkoitettu yksittäisille käyttäjille. Järjestelmä ei siis tue tiimityöskentelyä. Tämä tarkoittaa, että kukin projekti voi kuulua vain yhdelle käyttäjälle. Toisin sanoen projektit ja tehtävät ovat käyttäjäkohtaisia. Näitä tietoja ei voi jakaa muiden käyttäjien kanssa.

\begin{figure}[htb]
\begin{center}
\includegraphics[width=\textwidth]{img/kayttotapauskaavio.png}
\caption{Käyttötapauskaavio}
\label{kasitekaavio}
\end{center}
\end{figure}

\section{Järjestelmän tietosisältö}

Järjestelmän tietosisältöä kuvaavia käsitteitä ovat käyttäjä, projekti ja tehtävä. Järjestelmä toimii siten, että kullakin käyttäjällä on omat projektinsa ja tehtävänsä. Tässä järjestelmässä tehtävä on työn perusyksikkö, jolla on tietty päivämäärä. Tehtäviä voi luokitella jakamalla niitä projekteihin. Käyttäjällä voi olla mielivaltainen määrä projekteja, joihin voi liittyä mielivaltainen määrä tehtäviä. Kukin tehtävä voi kuulua yhteen tai useampaan projektiin.

\begin{figure}[htbp]
\begin{center}
\includegraphics[width=\textwidth]{img/kasitekaavio.pdf}
\caption{Käsitekaavio}
\label{kasitekaavio}
\end{center}
\end{figure}

\begin{table}[htbp]
\renewcommand{\arraystretch}{1.2}
\caption{Järjestelmän tietokohteiden kuvaukset}
\makebox[\textwidth][c]{
\begin{tabularx}{1.15\textwidth}{lllX}
\toprule
Käsite & Attribuutti & Arvojoukko & Kuvaus \\
\midrule
Person &
forename    & Merkkijono (max 50)  & Käyttäjän etunimi      \\ 
& surname     & Merkkijono (max 50)  & Käyttäjän sukunimi    \\ 
& email         & Merkkijono (max 30)  & Käyttäjän sähköpostiosoite   \\ 
& password    & Merkkijono (max 100) & Käyttäjän salasana \\
\midrule
Project &
name    & Merkkijono (max 100) & Projektin nimi \\
\midrule
Task &
name         & Merkkijono (max 250)  & Tehtävän nimi      \\ 
& priority      & Kokonaisluku            & Kokonaisluku välillä 1-4.    \\ 
& schedule     & Päivämäärä              & Suunniteltu suorituspäivä   \\ 
& completed   & Totuusarvo              & Kertoo onko tehtävä suoritettu \\
\bottomrule
\end{tabularx}
}
\end{table}

Käyttäjän salasana on merkkijono. Salasanat voitaisiin tallentaan tiivistettynä, mutta järjestelmän nykyinen versio tallentaa salasanat sellaisenaan puhtaana tekstinä.

Tehtävän prioriteetti on kokonaisluku välillä 1-4. Suurinta prioriteettia merkitään luvulla 1 ja pienintä prioriteettia luvulla 3. Jos prioriteettia ei ole asetettu lainkaan, sen arvoa merkitään luvulla 4. Prioriteetti on siis sitä suurempi, mitä pienempi sen lukuarvo on, ja asteikon nollakohta on luku 4.

\section{Relaatiotietokantakaavio}

Tietokantakaavio voidaan johtaa käsitekaaviosta lisäämällä kuhunkin tauluun pääavain, ja esittämällä yhteydet vierasavainten avulla. Projektin ja tehtävän välillä on monesta\hyp{}moneen\hyp{}tyyppinen yhteys, jonka esittämiseen tarvitaan lisäksi yksi välitaulu.

\begin{figure*}[htbp]
\begin{center}
\makebox[\textwidth][c]{\includegraphics[width=1.3\textwidth]{img/tietokantakaavio.pdf}}
\caption{Tietokantakaavio}
\label{tietokantakaavio}
\end{center}
\end{figure*}

\section{Käynnistys-/käyttöohje}

Sovellus on asennettu Herokun pilvipalveluun. Sovellus löytyy osoitteesta

\begin{center}\url{http://tlist.herokuapp.com}\end{center}

Järjestelmään voi kirjautua testitunnuksella \textbf{test@user.com}, jonka salasana on \textbf{pwd}.

Projekteja voi selata vasemman reunan sivupalkista. Projektiin kuuluvat tehtävät saa näkyville klikkaamalla projektin nimeä. Projektin tehtävien listauksessa on toiminnot myös tehtävien lisäämiseen ja muokkaamiseen. Nappi tehtävän luomiseen on tehtävälistauksen alapuolella. Tehtävän muokkaaminen ja lisääminen tapahtuvat erillisen dialogin kautta. Dialogissa on nappi myös tehtävän poistamiseen.

\end{document}


%%% Local Variables: 
%%% mode: latex
%%% TeX-master: t
%%% End: 
